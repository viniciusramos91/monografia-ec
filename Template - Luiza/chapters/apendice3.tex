\chapter{\textit{Queries} Gerais Utilizadas nas Consultas SQL}  \label{apendiceD}


\noindent \textbf{\textit{Query para geração da tabela \textit{evasao\_todos}, com dados dos alunos e do histórico:}}
\newline
\newline
\textbf{INSERT INTO} evasao\_todos (MatricAluno, AluSexo, AluDtNasc, AluEscola, \newline 
AluCotID, AluCotTipo, AluAnoIngresso, AluSemestreIngresso,
AluFormaIngresso,  \newline 
AluAnoSaida, AluSemestreSaida, AluFormaSaida, AluSaidaMotivo, \newline 
AluSituacao, DiscAno, DiscSemestre,
DiscCodDisc, DiscCreditos, \newline
DiscTurma, DiscMencao, MateriaCursada, NomeMateria) \newline 
\textbf{SELECT} a.MatricAluno, a.AluSexo, a.AluDtNasc, a.AluEscola, a.AluCotId, a.AluCotTipo, \newline a.AluAnoIngresso, a.AluSemestreIngresso,
a.AluFormaIngresso, a.AluAnoSaida, \newline
a.AluSemestreSaida, a.AluFormaSaida, a.AluSaidaMotivo, a.AluSituacao, a.DiscAno, \newline
a.DiscSemestre, a.DiscCodDisc, a.DiscCreditos, a.DiscTurma, a.DiscMencao, \newline
a.MateriaCursada, o.nomeDisciplina \newline
\textbf{FROM} alunos\_historico a \newline
\textbf{LEFT OUTER JOIN} obrigatorias o \textbf{ON} a.DiscCodDisc = o.CodDisciplina
\newline
\newline
\newline
\textbf{\textit{Query utilizada para consultar dados dos alunos do sexo masculino:}}
\newline
\newline
\noindent \textbf{SELECT} distinct e.*, \newline
m.QtdeC1, m.PrimeiraNotaC1, m.PrimeiraTurmaC1, 
m.UltimaNotaC1, m.UltimaTurmaC1, m.StatusMencaoC1, \newline
m.QtdeC2, m.PrimeiraNotaC2, m.PrimeiraTurmaC2,
m.UltimaNotaC2, m.UltimaTurmaC2, m.StatusMencaoC2, \newline
m.QtdeC3, m.PrimeiraNotaC3, m.PrimeiraTurmaC3, 
m.UltimaNotaC3, m.UltimaTurmaC3, m.StatusMencaoC3, \newline
f.QtdeF1, f.PrimeiraNotaF1, f.PrimeiraTurmaF1,
f.UltimaNotaF1, f.UltimaTurmaF1, \newline 
f.StatusMencaoF1, \newline
f.QtdeF2, f.PrimeiraNotaF2, f.PrimeiraTurmaF2,
f.UltimaNotaF2, f.UltimaTurmaF2, \newline
f.StatusMencaoF2, \newline
f.QtdeF3, f.PrimeiraNotaF3, f.PrimeiraTurmaF3,
f.UltimaNotaF3, f.UltimaTurmaF3,\newline 
f.StatusMencaoF3, \newline
o.AluSexo, o.AluDtNasc, o.AluEscola, o.AluCotTipo, o.AluAnoIngresso,  \newline
o.AluSemestreIngresso, o.AluFormaIngresso, 
o.AluAnoSaida, o.AluSemestreSaida, \newline o.SemestreCursoSaida,
o.AluSaidaMotivo, o.AluSituacao \newline
\textbf{FROM} evasao\_cic e 
\textbf{INNER JOIN} evasao\_obrigatorias o \textbf{ON} e.Matricula = o.MatricAluno \newline
\textbf{LEFT OUTER JOIN} evasao\_mat m \textbf{ON} e.Matricula = m.Matricula \newline
\textbf{LEFT OUTER JOIN} evasao\_fis f \textbf{ON} e.Matricula = f.Matricula \newline
\textbf{WHERE} o.AluSexo = 'M' \textbf{AND} o.AluAnoIngresso \textbf{BETWEEN} 2000 \textbf{AND} 2013
\newline
\newline
\newline
\textbf{\textit{Query utilizada para consultar dados dos alunos do sexo feminino:}} 
\newline
\newline
\noindent \textbf{SELECT} distinct e.*, \newline
m.QtdeC1, m.PrimeiraNotaC1, m.PrimeiraTurmaC1, 
m.UltimaNotaC1, m.UltimaTurmaC1, m.StatusMencaoC1, \newline
m.QtdeC2, m.PrimeiraNotaC2, m.PrimeiraTurmaC2,
m.UltimaNotaC2, m.UltimaTurmaC2, m.StatusMencaoC2, \newline
m.QtdeC3, m.PrimeiraNotaC3, m.PrimeiraTurmaC3, 
m.UltimaNotaC3, m.UltimaTurmaC3, m.StatusMencaoC3, \newline
f.QtdeF1, f.PrimeiraNotaF1, f.PrimeiraTurmaF1,
f.UltimaNotaF1, f.UltimaTurmaF1, \newline 
f.StatusMencaoF1, \newline
f.QtdeF2, f.PrimeiraNotaF2, f.PrimeiraTurmaF2,
f.UltimaNotaF2, f.UltimaTurmaF2, \newline
f.StatusMencaoF2, \newline
f.QtdeF3, f.PrimeiraNotaF3, f.PrimeiraTurmaF3,
f.UltimaNotaF3, f.UltimaTurmaF3,\newline 
f.StatusMencaoF3, \newline
o.AluSexo, o.AluDtNasc, o.AluEscola, o.AluCotTipo, o.AluAnoIngresso,  \newline
o.AluSemestreIngresso, o.AluFormaIngresso, 
o.AluAnoSaida, o.AluSemestreSaida, \newline o.SemestreCursoSaida,
o.AluSaidaMotivo, o.AluSituacao \newline
\textbf{FROM} evasao\_cic e 
\textbf{INNER JOIN} evasao\_obrigatorias o \textbf{ON} e.Matricula = o.MatricAluno \newline
\textbf{LEFT OUTER JOIN} evasao\_mat m \textbf{ON} e.Matricula = m.Matricula \newline
\textbf{LEFT OUTER JOIN} evasao\_fis f \textbf{ON} e.Matricula = f.Matricula \newline
\textbf{WHERE} o.AluSexo = 'F' \textbf{AND} o.AluAnoIngresso \textbf{BETWEEN} 2000 \textbf{AND} 2013
\newline
\newline
\newline
\textbf{\textit{Query utilizada para consultar a taxa de alunos reprovados pela terceira vez por disciplina:}} 

\begin{longtable}{C{3cm}C{3cm}C{3cm}}
\label{query-reprova3} \\
\caption{Combinação de valores na \textit{query} por linha.} \\
\hline
\textit{tabela} & \textit{nota} & \textit{quantidade}\\
\hline
\multirow{3}{*}{\textit{evasao\_cic}} & UltimaNotaCB & QtdeCB\\
 & UltimaNotaED & QtdeED\\
 & UltimaNotaPS & QtdePS\\ \hline
 \multirow{3}{*}{\textit{evasao\_mat}} & UltimaNotaC1 & QtdeC1\\
 & UltimaNotaC2 & QtdeC2\\
 & UltimaNotaC3 & QtdeC3\\ \hline
 \multirow{3}{*}{\textit{evasao\_fis}} & UltimaNotaF1 & QtdeF1\\
 & UltimaNotaF2 & QtdeF2\\
 & UltimaNotaF3 & QtdeF3\\ \hline
\end{longtable}  

\noindent \textbf{SELECT} t.* \newline
\textbf{FROM} \textit{tabela} t \textbf{INNER JOIN} alunos a
on t.Matricula = a.MatricAluno \newline
\textbf{WHERE} a.AnoIngresso \textbf{BETWEEN} 2000 \textbf{AND} 2013 \newline
\textbf{AND} \textit{quantidade} = 3 \newline
\textbf{AND} (\textit{nota} = 'TR' \textbf{OR} \textit{nota} = 'TJ' 
\textbf{OR} \textit{nota} = 'MI' \textbf{OR} \textit{nota} = 'II' \textbf{OR} \textit{nota} = 'SR') \newline
\textbf{ORDER BY} \textit{nota} \textbf{ASC}
\newline
\newline
\newline
\textbf{\textit{Query utilizada para consultar a primeira menção obtida pelo aluno por disciplina:}} 
\begin{longtable}{C{3cm}C{5cm}}
	\label{query-primeiranota} \\
	\caption{Combinação de valores na \textit{query} por linha.} \\
	\hline
	\textit{tabela} & \textit{nota}\\
	\hline
	\multirow{2}{*}{\textit{evasao\_cic}} & PrimeiraNotaCB\\
	& PrimeiraNotaED\\
	&PrimeiraNotaPS\\ \hline
	\multirow{2}{*}{\textit{evasao\_mat}} & PrimeiraNotaC1\\
	& PrimeiraNotaC2\\
	& PrimeiraNotaC3\\ \hline
	\multirow{2}{*}{\textit{evasao\_fis}} & PrimeiraNotaF1\\
	& PrimeiraNotaF2\\
	& PrimeiraNotaF3\\ \hline
\end{longtable}  

\noindent \textbf{SELECT DISTINCT} t.\textit{nota}, \textbf{COUNT}(t.\textit{nota}) \newline 
\textbf{FROM} \textit{tabela} t \textbf{INNER JOIN} alunos a
\textbf{ON} t.Matricula = a.MatricAluno \newline
\textbf{WHERE} a.AnoIngresso \textbf{BETWEEN} 2000 \textbf{AND} 2013 \textbf{AND} \textit{nota} \textbf{IS NOT NULL} \newline
\textbf{GROUP BY} \textit{nota} \newline
\textbf{ORDER BY} \textit{nota} \textbf{ASC} 
\newline
\newline
\newline
\textbf{\textit{Query utilizada para consultar a primeira menção obtida pelo aluno desligado por disciplina:}}
\begin{longtable}{C{3cm}C{5cm}}
	\label{query-segundanota} \\
	\caption{Combinação de valores na \textit{query} por linha.} \\
	\hline
	\textit{tabela} & \textit{nota}\\
	\hline
	\multirow{2}{*}{\textit{evasao\_cic}} & PrimeiraNotaCB\\
	& PrimeiraNotaED\\
	&PrimeiraNotaPS\\ \hline
	\multirow{2}{*}{\textit{evasao\_mat}} & PrimeiraNotaC1\\
	& PrimeiraNotaC2\\
	& PrimeiraNotaC3\\ \hline
	\multirow{2}{*}{\textit{evasao\_fis}} & PrimeiraNotaF1\\
	& PrimeiraNotaF2\\
	& PrimeiraNotaF3\\ \hline
\end{longtable}  

\noindent \textbf{SELECT DISTINCT} t.\textit{nota}, \textbf{COUNT}(t.\textit{nota}) \newline 
\textbf{FROM} \textit{tabela} t \textbf{INNER JOIN} alunos a
\textbf{ON} t.Matricula = a.MatricAluno \newline
\textbf{WHERE} a.AnoIngresso \textbf{BETWEEN} 2000 \textbf{AND} 2013 \textbf{AND} \textit{nota} \textbf{IS NOT NULL} \newline
\textbf{AND} a.AluSituacao = 'DESLIGADO' \newline
\textbf{GROUP BY} \textit{nota} \newline
\textbf{ORDER BY} \textit{nota} \textbf{ASC}
\newline
\newline
\newline
\textit{\textbf{Query utilizada para selecionar as matrículas e quantidade de  vezes que determinada disciplina foi cursada pelo aluno:}}
\begin{longtable}{C{3cm}C{5cm}}
	\label{query-vezescursada} \\
	\caption{Combinação de valores na \textit{query} por linha.} \\
	\hline
	\textit{tabela} & \textit{quantidade}\\
	\hline
	\multirow{2}{*}{\textit{evasao\_cic}} & QtdeCB\\
	& QtdeED\\
	& QtdePS\\ \hline
	\multirow{2}{*}{\textit{evasao\_mat}} & QtdeC1\\
	& QtdeC2\\
	& QtdeC3\\ \hline
	\multirow{2}{*}{\textit{evasao\_fis}} & QtdeF1\\
	& QtdeF2\\
	& QtdeF3\\ \hline
\end{longtable}  

\noindent \textbf{SELECT DISTINCT} t.Matricula, t.\textit{quantidade} \newline
\textbf{FROM} \textit{tabela} t \textbf{INNER JOIN} alunos a
\textbf{ON} e.Matricula = a.MatricAluno \newline
\textbf{WHERE} a.AnoIngresso \textbf{BETWEEN} 2000 \textbf{AND} 2013
\textbf{AND} \textit{quantidade} \textbf{IS NOT NULL}
\newline
\newline
\newline
\textit{\textbf{Query utilizada verificar a quantidade de reprovações e a quantidade de alunos reprovados em determinada disciplina:}}
\begin{longtable}{C{3cm}C{3cm}C{5cm}}
	\label{query-reprovacoes} \\
	\caption{Combinação de valores na \textit{query} por linha.} \\
	\hline
	\textit{tabela} & \textit{quantidade} & \textit{nota}\\
	\hline
	\multirow{2}{*}{\textit{evasao\_cic}} & QtdeCB & PrimeiraNotaCB\\
	& QtdeED & PrimeiraNotaED\\
	& QtdePS & PrimeiraNotaPS\\ \hline
	\multirow{2}{*}{\textit{evasao\_mat}} & QtdeC1 & PrimeiraNotaC1\\
	& QtdeC2 & PrimeiraNotaC2\\
	& QtdeC3 & PrimeiraNotaC3\\ \hline
	\multirow{2}{*}{\textit{evasao\_fis}} & QtdeF1 & PrimeiraNotaF1\\
	& QtdeF2 & PrimeiraNotaF2\\
	& QtdeF3 & PrimeiraNotaF3\\ \hline
\end{longtable}  

\noindent \textbf{SELECT} t.\textit{quantidade} as 'Numero de Reprovações', \textbf{COUNT}(t.\textit{quantidade}) \textbf{AS} 'Quantidade de Alunos' \newline
\textbf{FROM} \textit{tabela} t \textbf{INNER JOIN} alunos a
\textbf{ON} e.Matricula = a.MatricAluno \newline
\textbf{WHERE} a.AnoIngresso \textbf{BETWEEN} 2000 \textbf{AND} 2013 \newline
\textbf{AND} \textit{quantidade} > 1 \textbf{OR} (\textit{quantidade} = 1 \textbf{OR} \textit{nota} = 'TR' \textbf{OR} \textit{nota} = 'TJ' \newline \textbf{OR} 
\textit{nota} = 'SR' \textbf{OR} \textit{nota} = 'II' \textbf{OR} \textit{nota} = 'MI') \newline
\textbf{GROUP BY} \textit{quantidade} \newline
\textbf{ORDER BY} \textit{quantidade} \textbf{ASC} \newline
\newline
\newline
\newline
\textit{\textbf{Query utilizada para contar a quantidade de alunos que não cursaram determinada disciplina:}}

\begin{longtable}{C{3cm}C{5cm}}
	\label{query-reprovacoes2} \\
	\caption{Combinação de valores na \textit{query} por linha.} \\
	\hline
	\textit{tabela}  & \textit{nota}\\
	\hline
	\multirow{2}{*}{\textit{evasao\_cic}} &  PrimeiraNotaCB\\
	& PrimeiraNotaED\\
	& PrimeiraNotaPS\\ \hline
	\multirow{2}{*}{\textit{evasao\_mat}} & PrimeiraNotaC1\\
	& PrimeiraNotaC2\\
	& PrimeiraNotaC3\\ \hline
	\multirow{2}{*}{\textit{evasao\_fis}} & PrimeiraNotaF1\\
	& PrimeiraNotaF2\\
	& PrimeiraNotaF3\\ \hline
\end{longtable}  

\textbf{SELECT COUNT} (*) \newline
\textbf{FROM} \textit{tabela} t \textbf{INNER JOIN} alunos a
\textbf{ON} e.Matricula = a.MatricAluno \newline
\textbf{WHERE} a.AnoIngresso \textbf{BETWEEN} 2000 \textbf{AND} 2013
\textbf{AND} \textit{nota} \textbf{IS NULL} \newline
\newline
\newline
\newline
\textit{\textbf{Query utilizada para contar a quantidade de alunos desligados que não cursaram determinada disciplina:}}

\begin{longtable}{C{3cm}C{5cm}}
	\label{query-reprovacoes3} \\
	\caption{Combinação de valores na \textit{query} por linha.} \\
	\hline
	\textit{tabela}  & \textit{nota}\\
	\hline
	\multirow{2}{*}{\textit{evasao\_cic}} &  PrimeiraNotaCB\\
	& PrimeiraNotaED\\
	& PrimeiraNotaPS\\ \hline
	\multirow{2}{*}{\textit{evasao\_mat}} & PrimeiraNotaC1\\
	& PrimeiraNotaC2\\
	& PrimeiraNotaC3\\ \hline
	\multirow{2}{*}{\textit{evasao\_fis}} & PrimeiraNotaF1\\
	& PrimeiraNotaF2\\
	& PrimeiraNotaF3\\ \hline
\end{longtable}  

\noindent \textbf{SELECT COUNT} (*) \newline
\textbf{FROM} \textit{tabela} t \textbf{INNER JOIN} alunos a
\textbf{ON} e.Matricula = a.MatricAluno \newline
\textbf{WHERE} a.AnoIngresso \textbf{BETWEEN} 2000 \textbf{AND} 2013
\textbf{AND} \textit{nota} \textbf{IS NULL  \newline AND} o.AluSituacao = 'DESLIGADO' \newline
\newline
\newline
\newline
\textit{\textbf{Query utilizada para verificar a situação geral dos alunos do Departamento:}}
\newline
\newline
\noindent \textbf{SELECT DISTINCT} a.alusituacao \textbf{AS} 'Situação', \textbf{COUNT}(*) as 'Quantidade' \newline
\textbf{FROM} alunos a \textbf{INNER JOIN} \textit{evasao\_cic} c
\textbf{ON} a.MatricAluno = c.Matricula \newline
\textbf{WHERE} a.AnoIngresso  \textbf{BETWEEN} 2000 \textbf{AND} 2013 \newline
\textbf{GROUP BY} a.alusituacao \newline
\textbf{ORDER BY} a.aluSituacao \textbf{ASC} \newline