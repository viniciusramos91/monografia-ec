
\chapter{Código Java para geração das tabelas \textit{evasao\_cic, evasao\_mat} e \textit{evasao\_fis}} \label{apendiceB}


A mesma estrutura de código foi utilizada para a geração das tabelas \textit{evasao\_cic, evasao\_mat} e \textit{evasao\_fis}, onde apenas o nome da tabela e os códigos das disciplinas foram alterados, seguindo a ordem apresentada na Tabela \ref{tabela-apendice}. 

\begin{table}[!h]
	\centering
	\label{tabela-apendice}
	\caption{Tabelas SQL e Códigos das Disciplinas por Departamento.}
	\begin{tabular}{C{3cm}|C{2cm}C{6cm}}
		\hline
		Tabela SQL & Código & Disciplina\\ \hline
		\multirow{3}{*}{\textit{evasao\_cic}} & 116301 & Computação Básica\\
		& 116319 & Estruturas de Dados\\
		& 113956 & Programação Sistemática\\ \hline
		\multirow{3}{*}{\textit{evasao\_mat}} & 113034 & Cálculo 1\\
		& 113042 & Cálculo 2\\
		& 113051 & Cálculo 3\\ \hline
		\multirow{3}{*}{\textit{evasao\_fis}} & 118001 & Física 1\\
		& 118028 & Física 2\\
		& 118044 & Física 3\\ \hline
	\end{tabular}
\end{table}

A seguir, é apresentado o exemplo utilizado para a geração da tabela \textit{evasao\_cic}. \newline
\newline
\newline
package Dados; \newline
import java.sql.Connection; \newline
import java.sql.DriverManager; \newline
import java.sql.SQLException; \newline
\newline

public class Constroi\_Dados\_CIC \{ 
\begin{quote}
	private static java.sql.Statement stm; \newline
	private static java.sql.Statement stm2; \newline
	private static java.sql.PreparedStatement ps; \newline
	private static Connection conexao; \newline
	\newline
	//	Função: Estabelece conexão com o banco de dados. \newline
	public static Connection ObtemConexao() throws SQLException, ClassNotFoundException\{
		\begin{quote}
		Connection conn = null; \newline
		Class.forName("com.mysql.jdbc.Driver"); \newline
		conn = DriverManager.getConnection(''jdbc:mysql://localhost:3306/alunos\_cic'', ''root'', '' ''); \newline
		return conn;
		\end{quote}
	\}  
	
	\textit{// Função: Insere matricula e quantidade de vezes que um aluno cursou a disciplina de Computação Básica.} \newline
	public static void AtualizaCB() throws ClassNotFoundException, SQLException \{	
	\begin{quote}
		String sql = "select distinct MatricAluno, count(CodDisc) from historico where codDisc = 116301 " \newline
		+ "group by MatricAluno order by MatricAluno asc"; \newline
		conexao = ObtemConexao(); \newline
		stm = conexao.createStatement(); \newline
		java.sql.ResultSet Rs = stm.executeQuery(sql); \newline
		while (Rs.next())\{
		\begin{quote}
			int matricula = Rs.getInt(1); \newline
			int quantidade =  Rs.getInt(2); \newline
			String sqlAtualiza = "insert into evasao\_cic (matricula, QtdeCB) values ("+matricula+","+quantidade+") "; \newline
			ps = conexao.prepareStatement(sqlAtualiza); \newline
			ps.execute(); 
			\end{quote}
		\} \newline
		Rs.close(); \newline
		AtualizaED(); \textit{//Atualiza a quantidade de vezes que a disciplina Estruturas de Dados foi cursada.}\newline
		AtualizaPS(); \textit{//Atualiza a quantidade de vezes que a disciplina Programação Sistemática foi cursada.}\newline
		AtualizaMencoesCIC(); \textit{//Atualiza as menções e turmas de Computação Básica, Estruturas de Dados e Programação Sistemática.}
		\end{quote}
	\} \newline
	
	\textit{// Função: Atualiza a quantidade de vezes que a disciplina Estruturas de Dados foi cursada por um aluno.} \newline
	public static void AtualizaED() throws SQLException, ClassNotFoundException \{
	\begin{quote}
		String sql = "select distinct MatricAluno, count(CodDisc) from historico where codDisc = 116319" \newline
		+ "group by MatricAluno"; \newline
		java.sql.ResultSet Rs = stm.executeQuery(sql); \newline
		while (Rs.next()) \{
		\begin{quote}
			int matricula = Rs.getInt(1); \newline
			int quantidade =  Rs.getInt(2); \newline
			String sqlAtualiza = "update evasao\_cic set QtdeED =" + quantidade + " where matricula =" + matricula; \newline
			ps = conexao.prepareStatement(sqlAtualiza); \newline
			ps.execute();
		\end{quote}
		\} \newline
		Rs.close();
	\end{quote}
	\}\newline
	\newline
	\textit{// Função: Atualiza a quantidade de vezes que a disciplina Programação Sistemática foi cursada por um aluno.} \newline
	public static void AtualizaPS() throws SQLException, ClassNotFoundException \{
	\begin{quote}
		String sql = "select distinct MatricAluno, count(CodDisc) from historico where codDisc = 113956" \newline
		+ "group by MatricAluno"; \newline
		java.sql.ResultSet Rs = stm.executeQuery(sql); \newline
		while (Rs.next()) \{
		\begin{quote}
			int matricula = Rs.getInt(1); \newline
			int quantidade =  Rs.getInt(2); \newline
			String sqlAtualiza = "update evasao\_cic set QtdePS =" + quantidade + " where matricula =" + matricula; \newline
			ps = conexao.prepareStatement(sqlAtualiza); \newline
			ps.execute();
		\end{quote}
		\} \newline
		Rs.close();
	\end{quote}
	\}\newline
	\newline
	\textit{//	Atualiza as primeiras e últimas notas e turmas das disciplinas Computação Básica, Estruturas de Dados e Programação Sistemática para cada aluno}. \newline
	public static void AtualizaMencoesCIC() throws SQLException, ClassNotFoundException\{
	\begin{quote}
		conexao = ObtemConexao(); \newline
		stm2 = conexao.createStatement(); \newline
		String mencaoCB1; \newline
		String turmaCB1; \newline
		String mencaoED1; \newline
		String turmaED1; \newline
		String mencaoCB2; \newline
		String turmaCB2; \newline
		String mencaoED2; \newline
		String turmaED2; \newline
		String mencaoPS1; \newline
		String turmaPS1; \newline
		String mencaoPS2; \newline
		String turmaPS2; \newline
		String sql = "select Matricula from evasao\_cic"; \newline
		java.sql.ResultSet Rs5 = stm2.executeQuery(sql); \newline
		while (Rs5.next() == true)\{
		\begin{quote}
			int matricula = Rs5.getInt(1); \newline
			String sqlAtualizaCB1 = "select mencao, concat(ano,'\_',semestre,'\_',turma) from historico where codDisc=116301 and matricAluno =" + matricula + 
			" order by ano asc, semestre asc limit 1"; \newline
			java.sql.ResultSet Rs1 = stm.executeQuery(sqlAtualizaCB1); \newline
			if (Rs1.next())\{
			\begin{quote}
				mencaoCB1 = Rs1.getString(1); \newline
				turmaCB1 = Rs1.getString(2); 
			\end{quote}
			\} \newline
			else\{
			\begin{quote}
				mencaoCB1 = null; \newline
				turmaCB1 = null; \newline
				\} 
			\end{quote}
			String sqlAtualizaCB2 = "select mencao, concat(ano,'\_',semestre,'\_',turma) from historico where codDisc = 116301 and matricAluno ="
			 + matricula + " order by ano desc, semestre desc limit 1"; \newline
			
			java.sql.ResultSet Rs3 = stm.executeQuery(sqlAtualizaCB2); \newline
				if (Rs3.next())\{
				\begin{quote}
					mencaoCB2 = Rs3.getString(1); \newline
					turmaCB2 = Rs3.getString(2); 
				\end{quote}
				\} else \{
				\begin{quote}
				mencaoCB2 = null; \newline
				turmaCB2 = null; \newline
					\} 
				\end{quote}
			String sqlAtualizaED1 = "select mencao, concat(ano,'\_',semestre,'\_',turma) from historico where codDisc=116319 and matricAluno =" + matricula +
			" order by ano asc, semestre asc limit 1"; \newline
			java.sql.ResultSet Rs2= stm.executeQuery(sqlAtualizaED1); \newline
			if (Rs2.next())\{
			\begin{quote}
				mencaoED1 = Rs2.getString(1); \newline
				turmaED1 = Rs2.getString(2); 
			\end{quote}
			\}
			else\{ 
			\begin{quote}
				mencaoED1 = null; \newline
				turmaED1 = null; \newline
				\} 
			\end{quote}
			String sqlAtualizaED2 = "select mencao, concat(ano,'\_',semestre,'\_',turma) from historico where codDisc = 116319 and matricAluno ="
			 + matricula + " order by ano desc, semestre desc limit 1"; \newline
			java.sql.ResultSet Rs4 = stm.executeQuery(sqlAtualizaED2); \newline
			if (Rs4.next())\{ 
			\begin{quote}
				mencaoED2 = Rs4.getString(1); \newline
				turmaED2 = Rs4.getString(2); 
			\end{quote}
			\} else \{
			\begin{quote}
			mencaoED2 = null; \newline
			turmaED2 =null; \newline
			\} 
			\end{quote}
		String sqlAtualizaPS1 = "select mencao, concat(ano,'\_',semestre,'\_',turma) from historico where codDisc = 116319 and matricAluno =" + matricula + 
		" order by ano asc, semestre asc limit 1"; \newline
		java.sql.ResultSet Rs7= stm.executeQuery(sqlAtualizaPS1); \newline
		if (Rs7.next())\{
		\begin{quote}
			mencaoPS1 = Rs7.getString(1);\newline
			turmaPS1 = Rs7.getString(2);
		\end{quote}
		\} else\{
		\begin{quote}
			mencaoPS1 = null; \newline
			turmaPS1 = null; \newline
			\}
			\end{quote}
		
		String sqlAtualizaPS2 = "select mencao, concat(ano,'\_',semestre,'\_',turma) from historico where codDisc = 116319 and matricAluno ="
		 + matricula + " order by ano desc, semestre desc limit 1"; \newline
		java.sql.ResultSet Rs6 = stm.executeQuery(sqlAtualizaED2); \newline
		if (Rs6.next())\{
		\begin{quote}
			mencaoPS2 = Rs6.getString(1); \newline
			turmaPS2 = Rs6.getString(2);
		\end{quote}
		\} else \{
		\begin{quote}
		mencaoPS2 = null; \newline
		turmaPS2 =null; \newline
		\}
		\end{quote}
	String sqlAtualiza = "update evasao\_cic set PrimeiraNotaCB = '" + mencaoCB1 + "' , PrimeiraNotaED = '" + mencaoED1 + 
	"', PrimeiraNotaPS='" + mencaoPS1 +"', UltimaNotaCB ='" + mencaoCB2 + "', UltimaNotaED = '" + mencaoED2 + "', UltimaNotaPS = '" + mencaoPS2 + "',PrimeiraTurmaCB='" + turmaCB1 + 
	"', UltimaTurmaCB ='" + turmaCB2 + "', PrimeiraTurmaED='" + turmaED1 + "', UltimaTurmaED ='" + turmaED2 + "', PrimeiraTurmaPS ='" + 
	turmaPS1 + "', UltimaTurmaPS ='" + turmaPS2 +
	"' where matricula = "+matricula; \newline
	ps = conexao.prepareStatement(sqlAtualiza); \newline
	ps.execute(); 
	\end{quote}
	\} \newline
	Rs5.close();
\end{quote}
	\} \newline
	\newline
\textit{// Função Principal} \newline
 public static void main (String[] args) throws ClassNotFoundException, SQLException\{ 
 \begin{quote}
 	AtualizaCB(); 
 	\end{quote}
 \}
\end{quote}
\}