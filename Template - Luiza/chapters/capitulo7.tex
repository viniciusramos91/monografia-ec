

\chapter{Conclusão} \label{chapter7}

Neste trabalho, foram identificadas variáveis, regras e padrões capazes de auxiliar na detecção das causas de evasão dos alunos do Bacharelado em Ciência da Computação. Foram realizadas análises estatísticas e mineração de dados a partir dos dados disponibilizados sobre os alunos ingressantes no curso entre 2000 e 2013.

Foi verificado que, embora a taxa de ingresso tenha crescido ao longo dos anos, a quantidade de alunos que evadiram do curso é bastante superior à quantidade de alunos formados. O percentual de alunas formadas é superior ao de alunos formados, embora o ingresso de alunos seja muito predominante. A partir de 2008, é verificado um decréscimo na taxa de formados por semestre de ingresso, visto que a maioria dos alunos levam mais de 10 semestres para concluir o curso. Para os alunos (do sexo masculino), não foi identificado um período com maior índice de evasão, enquanto que para as alunas a evasão é maior entre o primeiro e o terceiro semestres do curso.

Foi verificado que o vestibular e o PAS constituem as principais formas de ingresso no curso. Os percentuais de ingresso e desligamento do vestibular são maiores se comparados aos do PAS. Não constatou-se, entre 2000 e 2013, alunos formados que ingressaram por matrícula cortesia, convênio internacional e transferência facultativa. Como a Universidade de Brasília adotou o ENEM como forma de ingresso a partir de 2012, ainda não foram constatados alunos formados. 

Entre as formas de desligamento, as principais apontadas foram desligamento por não cumprimento de condição, desligamento por abandono e desligamento voluntário. Foi verificado que os alunos  desligados em geral cursaram, em média, de cinco a seis semestres. Ainda, foi possível verificar para as alunas, uma maior taxa de desligamento entre o primeiro e o terceiro semestres, o que indica que as alunas tendem a ter evasão do curso principalmente nos dois primeiros anos de graduação.

Comparando a média de desempenho dos alunos em geral e dos alunos desligados, verificou-se que os últimos apresentaram maiores taxas de reprovação, se comparados à media geral. Constatou-se que os alunos apresentam maior dificuldade nas disciplinas ofertadas pelo Departamento de Matemática e o Instituto de Física, especialmente em Cálculo 1, Física 1, Cálculo 3 e Física 3, considerando que essas disciplinas apresentaram altas taxas de reprovação. Das disciplinas analisadas ofertadas pelo Departamento de Ciência da Computação, Computação Básica e Programação Sistemática são as disciplinas com maiores taxas de reprovação, porém os percentuais de reprovação são relativamente menores se comparados às disciplinas de Cálculo e Física.
 
Ao realizar a mineração de dados no Weka, os algoritmos de classificação apresentaram variação no desempenho para cada arquivo de dados analisado, sendo necessária a utilização de diferentes algoritmos para identificar padrões e similaridades entre as regras geradas. Durante a mineração de dados, houve algoritmos que, embora apresentassem uma alta taxa de acerto, não forneciam regras ou informações relevantes para justificar as causas de evasão.

Analisando os dados gerais, os algoritmos apontaram na maioria dos casos que a reprovação ou um baixo desempenho nas disciplinas de Cálculo 1, Física 1, Cálculo 3 e Física 3 estavam relacionados ao desligamento. Porém foram identificadas algumas exceções, como a condição do aluno ser aprovado em Cálculo 2 com menção SS e Física 2 com menção MS ser considerado desligado pelo algoritmo de regra \textit{PART}. Para a regra, foram verificadas seis instâncias na base de dados geral e três instâncias abrangidas pela regra no conjunto de teste no Weka. Ao verificar a base de dados, os alunos desligados que obtiveram menção SS em Cálculo 2 e MS em Física 2 evadiram principalmente por não cumprimento de condição e desligamento voluntário.

 Outra exceção gerada pelo algoritmo \textit{DecisionTable} é dada a aprovação em Cálculo 3 com menção SS e Física 3 com menção MM o aluno ser classificado como desligado, porém ao verificar a base de dados não foi constatada nenhuma instância abrangida pela regra. Também foi verificado que todos os algoritmos aplicados sobre o conjunto de dados gerais apontaram uma tendência dos alunos a saírem do curso antes da conclusão, visto que todos os algoritmos aplicados sobre o conjunto de dados gerais classificaram mais alunos que estavam cursando em 2014 como prováveis desligados do que formados.

Analisando os dados por semestre de ingresso, foi verificado que os atributos relacionados às turmas e menções foram fortemente utilizados para a geração de regras de classificação pelos algoritmos. Pelos algoritmos de classificação, verificou-se que as turmas nas quais as disciplinas foram cursadas também foram determinantes para classificar determinado aluno como cursando, formado ou desligado, principalmente em Cálculo 2, o que sugere a necessidade de um estudo aprofundado a respeito da relação entre as turmas e a continuidade do curso.  Das regras geradas, observou-se que as causas de evasão são variadas para cada semestre analisado, porém o baixo rendimento nas disciplinas de Cálculo 1, Cálculo 3, Física 1, Física 3, Estruturas de Dados e Programação Sistemática apresentaram-se como as causas de evasão mais recorrentes. 

\section{Contribuições} 

Os processos de mineração de dados e de análise estatística realizados neste trabalho permitiram validar a hipótese apresentada na Seção \ref{1title3}, que era identificar informações relevantes para relacionar padrões descobertos nos dados com o processo de desligamento dos alunos no Bacharelado em Ciência da Computação da Universidade de Brasília.

Em resumo, dos resultados obtidos nesse trabalho, sugerimos que o baixo rendimento nas disciplinas de Cálculo 1, Cálculo 3, Física 1, Física 3 e Programação Sistemática podem ser utilizadas como parâmetros para identificar um aluno com risco de evasão.

\section{Trabalhos Futuros}

Como sugestões para trabalhos futuros, são propostos:

\begin{itemize}

	\item Comparar o desempenho dos alunos matriculados no Bacharelado, Licenciatura e Engenharia da Computação nas disciplinas iniciais ofertadas aos três cursos, tais como Cálculo 1 e 2, Computação Básica e Estruturas de Dados;
	\item Comparar o desempenho dos alunos matriculados no Bacharelado em Ciência da Computação nas disciplinas finais de curso, tais como Organização e Arquitetura de Computadores, \textit{Software} Básico e Sistemas Operacionais;
	\item Analisar novos atributos (por exemplo, dados informados na inscrição do vestibular ou PAS) para identificar novos padrões nos perfis de alunos com risco de evasão;
	\item Repetir o experimento futuramente para avaliar se houve diminuição das taxas de evasão com a implantação do novo Projeto Político Pedagógico no Bacharelado em Ciência da Computação.
\end{itemize}
