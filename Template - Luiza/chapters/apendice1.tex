
\chapter{Código Java para cálculo da quantidade de semestres cursados pelos alunos} \label{apendiceA}


\noindent package Dados; \newline
import java.sql.Connection; \newline
import java.sql.DriverManager; \newline
import java.sql.SQLException; 
\newline
\newline
public class Corrige\_Semestre \{ 
\begin{quote}
	private static java.sql.Statement stm; \newline
	private static java.sql.Statement stm2; \newline
	private static java.sql.PreparedStatement pst; \newline
	private static Connection conexao; \newline
	\newline		
	// Função: Estabelece conexao com o banco de dados \newline
	public static Connection ObtemConexao() throws SQLException, ClassNotFoundException \{
	\begin{quote}
		Connection conn = null; \newline
		Class.forName ("com.mysql.jdbc.Driver"); \newline
		conn = DriverManager.getConnection(''jdbc:mysql://localhost:3306/alunos\_cic'', ''root'', '' '' ); \newline
		return conn;
	\end{quote}
	\} \newline
	\newline
	//	Função: Determina quantos semestres o aluno cursou até a saída do curso \newline
	public static void AtualizaSemestreCurso() throws ClassNotFoundException, SQLException \{
	\begin{quote}
		String sql1 = null; \newline
		int valor = 0; \newline
		String sql = "select MatricAluno, anoIngresso, SemestreIngresso, AnoSaida, SemestreSaida from alunos"; \newline
		conexao = ObtemConexao(); \newline
		stm = conexao.createStatement(); \newline
		java.sql.ResultSet Rs = stm.executeQuery(sql); \newline
		while (Rs.next()) \{
			\begin{quote}
				int matricula = Rs.getInt(1); \newline
				int anoIngresso = Rs.getInt(2); \newline
				int semestreIngresso =  Rs.getInt(3); \newline
				int anoSaida = Rs.getInt(4); \newline
				int semestreSaida = Rs.getInt(5); \newline
				if (anoSaida == 9999)\{
				\begin{quote}
					sql1 = "update alunos set semestreSaidaCurso = 100 where matricAluno = " + matricula;
				\end{quote}
				\} else \{
				\begin{quote}
					if (semestreSaida == 9) \{
					\begin{quote}
						sql1 = "update alunos set semestreSaidaCurso = 99 where matricAluno = " + matricula;
					\end{quote}
					\} else \{
					\begin{quote}
						if (semestreIngresso == 1) \{
						\begin{quote}
							valor = ((anoSaida - anoIngresso)*2);
						\end{quote}
						\} 
					\end{quote}
					\begin{quote}
						if (semestreIngresso == 2) \{
						\begin{quote}
							valor = ((anoSaida - anoIngresso)*2) - 1;
						\end{quote}
						\}
					\end{quote}
					\begin{quote}
						if ((semestreSaida == 1 )||(semestreSaida == 0))\{
						\begin{quote}
						valor = valor +1;
						\end{quote}
						\} 
					\end{quote}
					\begin{quote}
						if (semestreSaida == 2) \{
						\begin{quote}
							valor = valor +2;
						\end{quote}
						\} \newline
						sql1 = "update alunos set semestreSaidaCurso =" + valor + "where matricAluno = " + matricula;
					\end{quote}
					\}
				\end{quote}
				\} \newline
				System.out.println(sql1); \newline
				pst = conexao.prepareStatement(sql1); \newline
				pst.execute(); 
			\end{quote}
			\}
			
	\end{quote}
	\}	\newline
	\newline 
	// Método Principal \newline
	public static void main (String[] args) throws ClassNotFoundException, SQLException \{
	\begin{quote}
		AtualizaSemestreCurso();
	\end{quote}	
	\}		
\end{quote}
		\}
	
