\textual

\chapter{Introdução} \label{1title1}
	
Todo ser humano tem o direito a educação, que deve ser garantida pelo Estado. Na verdade, é a própria Constituição de 1988 \citep{constituicao} que a enuncia como direito de todos, dever do Estado e da família, com as funções de garantir a realização plena do ser humano, inseri-lo no contexto do Estado Democrático e qualificá-lo para o mundo do trabalho. A um só tempo, a educação representa tanto mecanismo de desenvolvimento pessoal do indivíduo, como da própria sociedade em que ele se insere~\citep{raposo2005}.

A educação é mais que uma simples aquisição de saber. Ela propicia o desenvolvimento e a participação política ativa. É imprescindível para o desenvolvimento político e econômico, para a democracia e para a igualdade social, trazendo sustentabilidade para uma sociedade que deseja evoluir.

\section{Evasão no Ensino}

Um grande problema enfrentado pelas Instituições de Ensino fundamental e médio, e também de nível superior, é a saída dos estudantes sem conclusão dos estudos. A admissão não assegura a inclusão efetiva desses estudantes e tampouco sua permanência, sendo necessárias medidas mais eficazes que garantam a finalização escolar e acadêmica.

A evasão no ensino é um fenômeno social complexo, definido como interrupção no ciclo de estudos~\citep{gaioso2005} e tornou-se um problema que vem preocupando as Instituições de Ensino em geral, sejam elas públicas ou particulares, pois a saída de alunos provoca graves consequências sociais, acadêmicas e econômicas~\citep{baggi_lopes2010}. A perda com a evasão dos alunos pode ser mensurável em termos quantitativos, mas em termos qualitativos, torna-se uma tarefa com maior grau de complexidade. Isso porque a saída do aluno, sem a consequente conclusão de seu curso, traz perdas implícitas. Para ~\citet[p. 1]{lobo2011}, a evasão representa \textit{“uma perda social, de recursos e de tempo de todos os envolvidos no processo de ensino, pois perdeu o aluno, seus professores, a instituição de ensino, o sistema de educação e toda a sociedade (ou seja, o País)”}.

A evasão é um problema que aflige as Instituições de Ensino em geral. Conforme o Resumo Técnico do Censo da Educação Superior de 2009, desenvolvido pelo Ministério da Educação em parceria com Instituto Nacional de Pesquisas Anísio Teixeira (MEC/INEP) \citep{censo_2009}, os índices no âmbito universitário são altos e vêm sendo uma realidade cada vez mais presente nas Instituições de Ensino Superior. Em 2007, o Plano Nacional de Educação (PNE) fixou o objetivo de diminuir a taxa de evasão de alunos do ensino superior \citep{dias2010}.

Segundo ~\citet{braga2003}, o estudo e a análise sobre a evasão no ensino superior brasileiro apresenta-se  como uma proposta relevante, mas em contrapartida há poucas pesquisas disponíveis, visto que os estudos desenvolvidos a respeito de evasão escolar no Brasil iniciaram-se a partir de 1980. Para os autores, a evasão pode ser resultada por dois aspectos diferentes, seja pela própria iniciativa do aluno em sair do sistema de ensino ou pela conjuntura de fatores escolares, econômicos e pessoais, em que os dois primeiros aspectos podem corroborar mais para uma exclusão do que para uma evasão propriamente dita. Essa é uma questão atual, que está em pauta e que se agrava cada vez mais, conforme veremos adiante neste trabalho. E a construção de um modelo que permita identificar as causas e reduzir os índices de evasão no ensino superior brasileiro ainda é uma incógnita.

Em algumas análises divulgadas sobre o ingresso e evasão no Ensino Superior no Brasil, é notório o baixo percentual de alunos formados nos cursos pertencentes à área de Exatas \citep{censo2013}, tais como Ciências, Matemática e cursos ligados à área de Computação (tais como Ciência da Computação, Sistemas de Informação, Processamento de Dados e Automação). No levantamento  realizado pelo Sindicato das Entidades Mantenedoras de Estabelecimentos de Ensino Superior no Estado de São Paulo (Semesp), divulgado pelo portal  \citet{portalg1}, foi apontado que nos cursos de Sistemas de Informação, a cada três estudantes ingressantes, um chega a concluir os estudos. No caso da Ciência da Computação, a cada quatro ingressantes, apenas um conclui o curso, o que representa uma taxa de evasão de 75\%. O estudo divulgado pela Associação Brasileira das Empresas de Tecnologia da Informação e Comunicação ~\citep{brasscom} apontou que em 2010 o índice de evasão nos cursos de Tecnologia da Informação no Brasil foi de 87\%. No mesmo estudo, foi projetado que até o ano de 2014 poderia haver um \textit{déficit} de 45 mil profissionais para a área, e até 2022 seriam necessários 900 mil novos profissionais. Um segundo estudo realizado pela Associação Brasileira das Empresas de Tecnologia da Informação e Comunicação (Brasscom) sobre matrículas e concluintes para os cursos relacionados à TIC em Instituições de Ensino Superior ~\citep{brasscom2} (levando em consideração o triênio 2007-2009), apontou que no Distrito Federal, o número de concluintes dos cursos relacionados à TIC representam apenas 12,5\% sobre os matriculados, tendo sido registrados 7.779 matriculados e apenas 970 concluintes.

Sob essa perspectiva, alguns estudos foram realizados no Departamento de Ciência da Computação da Universidade de Brasília (CIC-UnB), com o objetivo de avaliar dados e informações que possibilitariam caracterizar as informações gerais sobre os cursos ofertados, as disciplinas e o público atendido pelo Departamento, o que envolveu a análise das taxas de ingresso, reprovação e evasão dos alunos matriculados. Atualmente, o Departamento de Ciência da Computação é responsável pela oferta de disciplinas em quatro cursos superiores: Bacharelado em Ciência da Computação, Licenciatura em Computação, Bacharelado em Engenharia da Computação  e Bacharelado em Engenharia Mecatrônica. Destes estudos, foi detalhado que a taxa de evasão dos alunos do Bacharelado em Ciência da Computação era de 55,76\% \citep{palmeira_santos2014}, o que pode ser considerado um número preocupante, assim como a diferença de matrícula e formandos por gênero, em que os alunos do sexo masculino representavam a maior proporção do público atendido pelo curso.

\section{Justificativa} \label{1title4}

Tendo em vista a alta taxa de evasão dos cursos ofertados pelo Departamento de Ciência da Computação da Universidade de Brasília, principalmente para o Bacharelado em Ciência da Computação, torna-se necessário um estudo que permita avaliar os motivos que podem levar o aluno à evasão, e estabelecer critérios que permitam identificar quais alunos têm maior propensão a evadir do curso logo nos semestres iniciais. Esse estudo pode contribuir para que o Departamento, em conjunto com os coordenadores e docentes, possam estabelecer medidas a fim de evitar que esse percentual aumente ao longo dos semestres consecutivos, e assim possibilitar a formação de um número maior de alunos para a área de Computação na Universidade de Brasília.

Nos últimos anos, a evasão no Bacharelado em Ciência da Computação na Universidade de Brasília está sendo utilizada como objeto de estudo de alunos e professores, que buscam identificar fatores que possam acarretar a desistência do aluno, e que auxiliem na identificação de alunos que estejam ou venham a estar sob risco de evasão.
Em trabalhos anteriores, foram identificados que os quatro primeiros semestres do curso são os que apresentam maiores taxas de evasão. Partindo desse pressuposto, surge a proposta de aprofundar o estudo de causas ou variáveis que possam estar relacionadas ao processo de evasão nos períodos iniciais, e formular uma análise que permita identificar novos padrões nos dados do departamento e predizer qual perfil de aluno tem maior propensão a evadir do curso.

\section{Problema} \label{1title6}

Atualmente, não há variáveis ou informações que permitam identificar alunos em risco de evasão no Bacharelado em Ciência da Computação da Universidade de Brasília.

\section{Objetivos} \label{1title2}

O objetivo geral deste trabalho é analisar e identificar as variáveis e fatores que justifiquem ou que estejam relacionadas a evasão nos primeiros semestres do curso de Ciência da Computação da Universidade de Brasília.


Os objetivos específicos constituem em:
\begin{itemize}
	\item Selecionar e organizar os dados fornecidos para análise;
	\item Realizar experimentos com dados oficiais do Bacharelado em Ciência da Computação;
	\item Identificar variáveis e padrões que possam estar relacionados com o processo de evasão no Bacharelado em Ciência da Computação da Universidade de Brasília;
	\item Discutir os resultados obtidos.     
\end{itemize}


\section{Hipótese} \label{1title3}
É possível identificar, por meio da mineração de dados, padrões nas causas de evasão dos alunos nos primeiros semestres do Bacharelado em Ciência da Computação da Universidade de Brasília.


\section{Descrição dos Capítulos} \label{1title5}

O Capítulo \ref{2title} aborda a evasão no Ensino Superior, destacando as causas e implicações na gestão das Instituições de Ensino Superior, além de apresentar trabalhos relacionados à evasão nos cursos de Computação realizados no Brasil, incluindo análises no Departamento de Ciência da Computação da Universidade de Brasília.

O Capítulo \ref{chapter4} aborda os conceitos de dado, informação e conhecimento, e a importância dos sistemas de informação e a gestão do conhecimento para as organizações. O Capítulo \ref{chapter3} apresenta o conceito, os componentes e as técnicas de mineração de dados, além das etapas do processo de extração do conhecimento, conhecido como KDD. É apresentado o \textit{software} Weka, utilizado na etapa de mineração de dados e algumas das funcionalidades disponíveis. 

O Capítulo \ref{chapter5} descreve a metodologia adotada para a análise e mineração dos dados, baseado nas etapas do processo KDD. Os resultados obtidos da análise estatística e mineração de dados são apresentados no Capítulo \ref{chapter6}, e o Capítulo \ref{chapter7} apresenta as conclusões obtidas e sugestões para trabalhos futuros.





	

